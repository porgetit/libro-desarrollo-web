\documentclass[12pt, letterpaper]{article}
\usepackage[utf8]{inputenc}

\title{La historia de internet: borrador 1}
\author{Kevin  Esguerra Cardona}
\date{Junio 2 del 2022}

\begin{document}

\maketitle

- Internet, que extraña palabra, ¿verdad?. Bueno, pues la verdad es que es de lo más sencilla de definir pero a su vez, una de las difíciles de dimensionar.
Verás, Internet significa, literalmente, red de conexiones lo cuál hace referencia a un conjunto de aparatos conectados todos juntos. Estos son los computadores.
Internet fue concebida por primera vez bajo el nombre de Arpanet, la cual era una red privada que conectaba algunas computadoras entre los estados de Utah y California
en Estados Unidos. En ese entonces solo conectaba un par de decenas de máquinas, algo muy impresionante para el época, pero que hoy en día lo puede hacer hasta un estudiante
de secundaria que tenga la sufiente dedicación y los recursos necesarios. Computadoras. Hoy en día, Internet conecta a millones de millones de computadoras, 
lo cuál nos brinda acceso a información de sobra suficiente para más de una vida.

- ¿Suena misterioso, verdad? es decir, ¿qué clase de "sistema" sería capaz de soportar tal cantidad de información sin colapsar?

- En realidad, es bastante sencillo, al menos en concepto; la idea es la siguiente: Imagina que tienes dos computadoras y deseas compartir información del 
dispositivo B al dispositivo A. Bueno pues, lo primero que seguramente se te ocurriría es usar una memoria USB, y es una buena idea, solo que tiene un gran pero, las memorias o
discos de almacenamiento no suelen tener una gran capacidad y ni hablar de la velocidad de transferencia. Demos por descartada esa opción, ya que necesitamos mover un Petabyte
de datos de video, por ejemplo. Ahora bien, que te parece mover el disco duro del PC B al PC A, esa sería una forma de transferir la información de un lugar al otro, pero ahora
imagina que necesitamos un programa en específico para edición de video que solo se puede instalar dentro del disco duro del PC A, por lo que al cambiar los discos, realmente 
nos quedaríamos donde comenzamos. Con esto en mente, ahora imagina que tienes un cable, un cable con la capacidad de transportar la información que va de un punto al otro. 
Eso es un cable de red. Pues, para solucionar tu problema, podrías conectar este cable en ambos dispositivos, el PC A y el PC B. Con esto, y un poco de "magia" por parte de la 
ingenieria de redes, podrías tener interconectados los sistemas operativos de ambos PCs y los discos duros de los mismos. Permitiendote así poder trabajar con la información de 
video almacenada en un sitio desde un PC diferente. Eso es internet, en su más simple escencia claro, la realidad es que internet es una red infinitamente más grande y compleja, 
pero que funciona en base a esa idea, la idea de un par de equipos que comparten al acceso a sus discos de almacenamiento.

- ¿Quién inventó el internet?

- Bueno, decirte que el internet lo inventó una sola persona sería ser un descarado mentiroso. Realmente, el internet es un conjunto de ideas de un sin número de visionarios
que imaginaron un mundo conectado entre sí. La idea del internet es tener acceso y comportir la información. Aunque, si debo declarar algunos de los científicos que más aportaron
debría indicarte a \textbf{Leonard Kleinrock} el cuál inventó la tecnología básica de internet, \textbf{Ray Tomlinson} que introdujo la mensajería electrónica y a 
\textbf{Tim Bernes-Lee} quién desarrolló el HTML y el sistema WWW.

- Pero, ¿porqué? ¿con qué objetivo se desarrolló tal tecnología?

- Eso es algo realmente simple, sigue el flujo de todos los grandes inventos de la humanidad, los más grandes, los que verdaderamente despiertan el interés de las personas o el
temor, en muchos casos. Pués resulta que el internet se desarrolló con fines de guerra. Arpanet, era específicamente un sistema de transmición de información militar, ideado
para permitir la comunicación entre mandos en caso de ataque. Ten en cuenta que esto fue en la época de la guerra fría entre Estados Unidos y la antigua Unión Soviética. Así
pues, internet era un seguro no se pensó para ser mucho más útil que un sistema de correos. Pero, mira lo que tenemos ahora, internet es "público", se estima que el 57\% de la
población mundial cuenta con acceso a internet en al menos un dispositivo.

Gracias a internet tenemos maravillas de la ingeniería cómo los coches que se conducen solos, los sistemas de GPS son en parte útiles gracias a grancantidad de información
proveniente de internet, las escuelas, los hospitales, los cuerpos de policia, prácticamente todo a donde miremos dentro de las grandes ciudades del siglo XXI está concetado a
internet. 

Ahora bien, cómo podrás imaginar, internet ha traído cambios muy significativos a nuestra sociedad. Muchos empleos que eran de suma importancia en antaño hoy no existen, por
ejemplo, los operadores telefónicos. Y esta tendencia continua, hoy vemos cómo trabajos tan espciealizados cómo la contaduría se ven afectados por el auge de tecnologías que
nativas de internet que tiene la capacidad de reemplazarlos. Internet es bueno, pero sus grandes avances hacen que a las personas les sea difícil adaptarse. Por eso, es ideal
que a las personas se les instruya en esta materia, que aprendan a trabajar con y para el internet.

Pero bueno, cómo imaginarás, internet no es solo una red de computadoras interconectadas. Quizá la parte más importante de la red es esa masa ingente de personas que día a día
buscan contribuir, que construyen el internet, desarrolladores web. La web o WWW (World Wide Web) es sistema de información que funciona a través de internet. La idea es que
con las herramientas adecuadas cualquier persona pueda compartir datos por medio de internet. Veamos un poco más acerca de la web en el siguiente capítulo.

$\newline$
Siguiente capítulo $\rightarrow$

\end{document}
