\documentclass{article}
\usepackage[utf8]{inputenc}

\title{Introducción: borrador 1}
\author{Kevin  Esguerra Cardona}
\date{Junio 1 del 2022}

\begin{document}

\maketitle

\textbf{Introducción}

A lo largo del siglo XX se dieron un sin fin de innovaciones que han moldeado nuestra sociedad hacia lo que tenemos hoy en día. Un de estos,
y a mi opinión el más importante, es el desarrollo de internet y por consiguiente, la World Wide Web. En este texto usted podrá aprender 
sobre la historia de todos aquellos acontecimientos que han logrado que nuestra sociedad se vea envuelta en una red informática tan profunda
y basta cómo el internet de la actualidad. Entenderá cómo funcionan las páginas web del lado del cliente y del lado del servidor. Conocerá
el funcionamiento de herramientas que le resultarán muy útiles para su trabajo cómo desarrollador. Podrá comprender el funcionamiento
práctico de los almacenes de datos de los que se surten los sitios web que usted visita diariamente. Y por último se sorprenderá con lo
mucho que ha avanzado el desarrollo web por medio de herramientas de desarrollo llamadas Frameworks.

Esta es una lista de todos los temas tratados en este libro: 
\begin{enumerate}
    \item Historia y evolución del internet.
    \item HTML5
    \item CSS3
    \item JavaScript
    \item Terminal de Windows
    \item Git y Github
    \item PHP
    \item Bases de datos
    \item Python
    \item El framework Django
\end{enumerate}

Cómo verá, es numerosa la lista de temas a tratar dentro de este curso de desarrollo web para principiantes. Por lo cuál, se ha tomado al decisión de usar una metodología de proyectos prácticos. Proyectos por medio de los cuales usted podrá poner a prueba los conocimientos recibidos en estos textos. La idea es la siguiente: usted verá una parte de teoría, la cuál irá en compañía de ejercicios que le facilitarán la comprensión del tema, así en cada nuevo tema. Al finalizar cada módulo se darán por comprendidos todos los temas tratados por medio de un proyecto final para el cual usted tendrá que utilizar todos los conocimientos adquiridos a lo largo del módulo. Un módulo está compuesto de temas. Estos proyectos usted podrá compartirlos en un foro a fin de recibir acompañamiento en caso de que lo requiera. El enlace a dicho foro es: (*Inserte el link al foro en Discord)

Este libro busca, a demás de ser un buen punto de partida para usted cómo futuro desarrollador, ser una guía a la cuál pueda acudir de forma rápida y eficiente a fin de recordar conceptos que se pueden escapar fácilmente a la retención de la memoria humana. Busca ser una especie de diccionario para el gran desarrollador web en el que usted se convertirá.

Para esto usted contará con ayudas cómo el ya mencionado "foro" en el cuál le sugiero interactuar a fin de aclarar todas las dudas que tenga. (*El foro puede ser en un servidor de Discord), también se le brindarán páginas de definiciones dónde podrá encontrar de forma clara el significado de todas aquellas palabras que puedan resultarle ajenas.

Con esto claro, sería justo decir que el camino no es sencillo. Las tecnologías que hoy intento explicar están bajo una constante actualización y puede que su comportamiento y modo de uso se vea modificado en el futuro. Por lo cuál, no me es posible escribir una guía definitiva para todo lo aquí contenido.

Dicho esto, es mi deber para conmigo, decirle que le estoy profundamente agradecido por permitirme guiarle en este bello, aunque tormentoso, camino al mundo del desarrollo web.

\end{document}
