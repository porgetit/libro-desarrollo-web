\documentclass{article}
\usepackage[utf8]{inputenc}

\title{Tesis del Libro: ideas varias.}
\author{Kevin  Esguerra Cardona}
\date{Mayo 28 del 2022}

\begin{document}

\maketitle

\section{Curso práctico de desarrollo web para mortales.}
La idea de este curso es la de guiar al lector a alcanzar los conocimientos básicos del desarrollo web de forma práctica.
Para esto, se implementará una metodología de "proyectos". Y, un solo proyecto final que condense todos los conocimientos brindados a lo
largo del libro. Cómo parte del contenido principal del mismo, se propone lo siguiente:
\begin{enumerate}
    \item Prefacio: En esta introducción se le hablará al lector sobre lo que aprenderá en los capítulos posteriores, así cómo del objetivo
    principal del libro. También se le compartirá el acceso a un foro en el cuál podrá solicitar ayuda en caso de necesitarla durante la
    realización del curso. 
    \item Historia del internet: Un texto corto que le enseñe al lector los orígenes de la red global que envuelve nuestra sociedad. 
    \item Tecnologías web: Desde el pasado al presente. \newline En este capítulo se buscará que el lector visualice todas las etapas por
    las que a atravesado el desarrollo de la "web" desde el más primitivo hipertexto hasta los modernos y automatizados frameworks cómo:
    Django y Laravel, entre otros muchos. 
    \item Preparación: En este capítulo se le indicará al lector cómo instalar las herramientas y dónde estará el material necesario para la
    correcta realización del curso. (Sería ideal indicar este proceso en al menos dos S.O diferentes p.e. Windows y Ubuntu/Windows y MacOS)
    \item Proyecto 1: En este capítulo el lector podrá aprender sobre tecnologías base de la web: HTML5, CSS3 y JavaScript. 
    Esto, a fin de que pueda adquirir las bases para avanzar y abstraer los conceptos fundamentales de tecnologías más modernas. 
    Durante este capitulo se indicaran los contenidos por medio de un proyecto práctico: Un blog personal.
    \item Introducción a la terminal y git: En este capítulo se le enseñará al lector lo básico sobre el manejo de la terminal, a demás se
    le enseñará sobre el servicio de GitHub y su herramienta git de tipo CLI.
    \item Proyecto 2: En este capítulo el lector podrá aprender a utilizar tecnologías de almacenamiento/repositorios con el objetivo de 
    que pueda guardar su progreso y compartirlo con la comunidad. Para este proyecto al lector construirá una galería fotográfica. 
    A demás, éste tendrá que almacenar su proyecto en un repositorio de github.com. 
    (Podría añadirse el contenido necesario para que el lector aprenda a "publicar" su sitio web en servicios cómo: Heroku, 000webhost.com,
    etc.)
    \item Un poco de bases de datos: En este capítulo se le enseñará al lector lo básico sobre bases de datos relacionales, de manera
    pragmática, a fin de que le sea posible comprender algunas partes de capítulos posteriores. 
    \item Una breve introducción a Python: Partiendo de el hecho de que le lector ahora comprende los principios básicos de la programación
    por medio del proyecto 1. Se le enseñará al lector la sintaxis básica de Python, así cómo todos aquellos detalles que se hagan
    indispensables para el correcto desarrollo de la siguiente parte del curso. (principalmente el manejo de módulos)
    \item Una pequeña introducción a Django: En este capítulo se le mostrará al lector que es Django, un poco de su historia, 
    sus capacidades y usos actuales. Así cómo también se le dará la indicación de cómo instalarlo por medio del gestor de librerías (pip) 
    de Python.
    \item Proyecto 3: En este capítulo el lector tendrá la posibilidad de construir un portafolio personal con su respectivo sitio de
    administración para crear, actualizar, revisar o eliminar publicaciones. Todo esto por medio del framework de desarrollo web "Django". En este proyecto se verán involucrados conocimientos de todos los anteriores capítulos, por lo cuál, actuará cómo capítulo final. 
    \item Cierre del curso: En este capítulo se le darán las gracias al lector por haber leído el libro. Así cómo también se felicitará a
    todos aquel que haya completad el curso satisfactoriamente. A demás, se le brindará acceso al lector a un foro, en el que podrá
    encontrar la ayuda que le sea necesaria (esto del foro debería ponerlo en la introducción del libro); y un repositorio de github en el
    que podrá encontrar todo el material, códigos y demás documentos utilizados para la realización de cada uno de los proyectos. 
\end{enumerate}

Cabe resaltar que todos los proyectos involucran el diseño web Responsive. 

\end{document}
