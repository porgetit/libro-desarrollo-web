\documentclass[12pt, letterpaper]{article}
\usepackage[utf8]{inputenc}

\title{Contenido del Libro: Borrador 1}
\author{Kevin  Esguerra Cardona}
\date{Mayo 30 del 2022}

\begin{document}

\maketitle

\begin{enumerate}
    \item Prefacio
    \item Introducción
    \item Parte 1: Antecedentes
    \begin{itemize}
        \item La historia del internet
        \item La evolución del desarrollo web
        \item Previas y programas necesarios (instalación) \newline*Solo se escribirá  una vez este el resto del curso completo
    \end{itemize}
    \item Parte 2: El diseño web
    \begin{itemize}
        \item Fundamentos de HTML5
        \begin{enumerate}
            \item ¿Qué es HTML? y ¿Por qué HTML5?
            \item Partes de un documento HTML
            \item HTML
            \begin{itemize}
                \item Elementos
                \item Atributos
                \item Encabezados
                \item Párrafos
                \item Citas
                \item Comentarios
                \item Enlaces
                \item Imágenes
                \item Favicon
                \item Tablas
                \item Listas
                \item Block e Inline
                \item Clases
                \item Identificadores
                \item Iframes
                \item Rutas
                \item head
                \item Elementos de diseño
                \item code
                \item Elementos semánticos
                \item Guía de estilo HTML
                \item Charset
                \item URL
                \item Formularios
                \item HTML Media
            \end{itemize}
        \end{enumerate}
        \item Fundamentos de CSS3.
        \begin{enumerate}
            \item ¿Qué es CSS? y ¿Por qué CSS3?
            \item CSS3
            \begin{itemize}
                \item Sintaxis
                \item Unidades
                \item Selectores
                \item Métodos de aplicación
                \item Comentarios
                \item Colores
                \item Fondos
                \item Bordes
                \item Márgenes
                \item Rellenos
                \item Anchos y alto
                \item Modelo de caja
                \item Esquema
                \item Textos
                \item Fuentes
                \item Íconos
                \item Enlaces
                \item Listas
                \item Tablas
                \item Display
                \item Ancho máximo
                \item Posicionamientos
                \item Índice Z
                \item Desbordamiento
                \item Flotación
                \item Inline-block
                \item Alineación
                \item Combinadores
                \item Pseudo-Clases
                \item Pseudo-Elementos
                \item Opacidad
                \item ¿Cómo hacer una barra de navegación? *¿?
                \item Objetos desplegables
                \item Selectores de atributos
                \item Formularios
                \item Contadores
                \item Diseño web
                \item Especificidad
                \item Regla !important
                \item Funciones matemáticas
                \item Esquinas redondeadas
                \item Imágenes con borde
                \item Gradientes
                \item Sombras
                \item Efectos de texto
                \item Transformaciones
                \item Transiciones
                \item Animaciones
                \item Tooltips
                \item Estilos para imágenes
                \item Reflección de imágenes
                \item Masking
                \item Botones
                \item Paginación
                \item Interfaz de usuario
                \item Media Queries
                \item Flexbox
                \item Grid
                \item Responsive Web Design
            \end{itemize}
        \end{enumerate}        
        \item Proyecto práctico 1.
    \end{itemize}
    \item Parte 3: La web interactiva
    \begin{itemize}
        \item Fundamentos de JavaScript
        \begin{enumerate}
            \item ¿Qué es JavaScript? y ¿Por qué ES6?
            \item JavaScript
            \begin{itemize}
                \item JS Where To
                \item JS Output
                \item JS Statements
                \item JS Comments
                \item JS Variables
                \item JS Let
                \item JS Const
                \item JS Operators
                \item JS Arithmetic
                \item JS Assignments
                \item JS Data Types
                \item JS Funtions
                \item JS Objects
                \item JS Events
                \item JS Strings
                \item JS String Methods
                \item JS String Search
                \item JS String Templates
                \item JS Numbers
                \item JS Number Methods
                \item JS Arrays
                \item JS Array Methods
                \item JS Array Iteration
                \item JS Array Const
                \item JS Dates
                \item JS Date Formats
                \item JS Date Get Methods
                \item JS Date Set Methods
                \item JS Math
                \item JS Random
                \item JS Booleans
                \item JS Comparisons
                \item JS If Else
                \item JS Switch
                \item JS Loop For
                \item JS Loop For In
                \item JS Loop For Of
                \item JS Loop While
                \item JS Break
                \item JS Iterables
                \item JS Sets
                \item JS Maps
                \item JS Typeof
                \item JS Type Conversion
                \item JS RegExp
                \item JS Errors
                \item JS Scope
                \item JS Hoisting
                \item JS Strict Mode
                \item JS this Keyword
                \item JS Arrow Function
                \item JS Classes
                \item JS Modules
                \item JS Json
                \item JS Debugging
                \item JS Style Guide
                \item JS Best Practices
                \item JS Mistakes
                \item JS Performance
                \item JS Reserved Words
                \item JS Async
                \begin{itemize}
                    \item JS Callbacks
                    \item JS Asynchronous
                    \item JS Promises
                    \item JS Async/Await
                \end{itemize}
                \item JS HTML DOM
                \begin{itemize}
                    \item JS DOM Methods
                    \item JS DOM Document
                    \item JS DOM Elements
                    \item JS DOM HTML
                    \item JS DOM Forms
                    \item JS DOM CSS
                    \item JS DOM Animations
                    \item JS DOM Events
                    \item JS DOM Event Listener
                    \item JS DOM Navigation
                    \item JS DOM Nodes
                    \item JS DOM Collector
                    \item JS DOM Node Lists
                \end{itemize}
            \end{itemize}
        \end{enumerate}
        \item Proyecto práctico 2
    \end{itemize}
    \item Parte 4: Las bases de datos
    \begin{itemize}
        \item ¿Qué es una base de datos?
        \item Fundamentos de Bases de Datos
        \item Fundamentos PHP
        \item Fundamentos de la terminal (Windows)
        \item Git y Github
        \item Proyecto práctico 3
    \end{itemize}
    \item Parte 5: Los frameworks
    \begin{itemize}
        \item ¿Qué es una framework?
        \item Fundamentos de Python
        \item Fundamentos de Django
        \item Proyecto práctico 4
    \end{itemize}
    \item Cierre del curso.
\end{enumerate}

\end{document}
