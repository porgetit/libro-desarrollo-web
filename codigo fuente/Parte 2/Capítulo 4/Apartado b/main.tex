\documentclass[12pt, letterpaper]{article}
\usepackage[utf8]{inputenc}
\usepackage{listings}
\usepackage{xcolor}

\definecolor{codegreen}{rgb}{0,0.6,0}
\definecolor{codegray}{rgb}{0.5,0.5,0.5}
\definecolor{codepurple}{rgb}{0.58,0,0.82}

\lstdefinestyle{mystyle}{
    backgroundcolor=\color{white},   
    commentstyle=\color{codegray},
    keywordstyle=\color{codegreen},
    numberstyle=\tiny\color{black},
    stringstyle=\color{codepurple},
    basicstyle=\ttfamily\footnotesize,
    breakatwhitespace=false,         
    breaklines=true,                 
    captionpos=b,                    
    keepspaces=true,                 
    numbers=left,                    
    numbersep=10pt,                  
    showspaces=false,                
    showstringspaces=false,
    showtabs=false,                  
    tabsize=2,
    frame=single
}

\lstset{literate=
  {á}{{\'a}}1 {é}{{\'e}}1 {í}{{\'i}}1 {ó}{{\'o}}1 {ú}{{\'u}}1
  {Á}{{\'A}}1 {É}{{\'E}}1 {Í}{{\'I}}1 {Ó}{{\'O}}1 {Ú}{{\'U}}1
  {à}{{\`a}}1 {è}{{\`e}}1 {ì}{{\`i}}1 {ò}{{\`o}}1 {ù}{{\`u}}1
  {À}{{\`A}}1 {È}{{\'E}}1 {Ì}{{\`I}}1 {Ò}{{\`O}}1 {Ù}{{\`U}}1
  {ä}{{\"a}}1 {ë}{{\"e}}1 {ï}{{\"i}}1 {ö}{{\"o}}1 {ü}{{\"u}}1
  {Ä}{{\"A}}1 {Ë}{{\"E}}1 {Ï}{{\"I}}1 {Ö}{{\"O}}1 {Ü}{{\"U}}1
  {â}{{\^a}}1 {ê}{{\^e}}1 {î}{{\^i}}1 {ô}{{\^o}}1 {û}{{\^u}}1
  {Â}{{\^A}}1 {Ê}{{\^E}}1 {Î}{{\^I}}1 {Ô}{{\^O}}1 {Û}{{\^U}}1
  {ã}{{\~a}}1 {ẽ}{{\~e}}1 {ĩ}{{\~i}}1 {õ}{{\~o}}1 {ũ}{{\~u}}1
  {Ã}{{\~A}}1 {Ẽ}{{\~E}}1 {Ĩ}{{\~I}}1 {Õ}{{\~O}}1 {Ũ}{{\~U}}1
  {œ}{{\oe}}1 {Œ}{{\OE}}1 {æ}{{\ae}}1 {Æ}{{\AE}}1 {ß}{{\ss}}1
  {ű}{{\H{u}}}1 {Ű}{{\H{U}}}1 {ő}{{\H{o}}}1 {Ő}{{\H{O}}}1
  {ç}{{\c c}}1 {Ç}{{\c C}}1 {ø}{{\o}}1 {å}{{\r a}}1 {Å}{{\r A}}1
  {€}{{\euro}}1 {£}{{\pounds}}1 {«}{{\guillemotleft}}1
  {»}{{\guillemotright}}1 {ñ}{{\~n}}1 {Ñ}{{\~N}}1 {¿}{{?`}}1 {¡}{{!`}}1,
  style=mystyle
}

\title{Partes de un documento HTML5}
\author{Kevin Esguerra Cardona}
\date{Junio 9 del 2022}

\begin{document}

\maketitle

\lstinputlisting[language=HTML]{index.html}

Dentro de todo documento HTML5 deben existir algunas líneas ¡siempre!, esas líneas son las que podemos ver en el recuadro de arriba. Voy a explicar lo que hace cada una de esas
líneas de código. 

\begin{enumerate}
    \item \textbf{Línea 1}: En esta línea se haya una etiqueta de las más nuevas. La etiqueta Doctype no es sensible a mayúsculas, por lo que da igual cómo la escribas, con que
    la escribas bien es suficiente. Este etiqueta viene desde html 4 y su función es la de brindar información al navegador acerca del tipo de documento que esta leyendo.
    \item \textbf{Línea 2}: En esta línea vemos a una etiqueta conocida \newline \verb|<html lang="es">| \newline esta etiqueta se encarga de abrir el documento HTML, su nombre correcto es etiqueta 
    HTML Root/Raíz. Dentro tiene el atributo lang, con el valor "es". Esta configuración de 
    la etiqueta define el idioma del sitio web en español.
    \item \textbf{Línea 3}: En esta línea tenemos a la etiqueta \verb|<head>| esta etiqueta define el inicio de la \textbf{cabecera del documento}. Un poco más adelante podrá ver lo 
    que es la cabecera del documento.
    \item \textbf{Línea 4}: En esta línea tenemos la etiqueta de título. Esta etiqueta define el título que se debe mostrar en la parte superior de la ventana del navegador, 
    en las pestañas de navegación.
    \item \textbf{Línea 5}: Esta es la etiqueta de cierre de la etiqueta \verb|<head>|, \verb|</head>|. La mayoría de etiquetas html cuentan con una estructura de caja, la cuál las condiciona a tener
    un \textbf{par apertura-cierre}.
    \item \textbf{Líneas 6-8}: En estas tres líneas se encuentra la estructura principal de un documento HTML, esta es conocida cómo el cuerpo del documento, body en inglés. El objetivo de esta etiqueta
    es servir de contenedor para todos los elementos que serán visualizados por el usuario a través del navegador. En la \textbf{línea 7} hay un comentario, algo de lo que aprenderemos más adelante. Por
    último, la \textbf{línea 8} contiene la etiqueta de cierre del elemento body.
    \item \textbf{Línea 9}: Está línea contiene la etiqueta de cierre del elemento HTML, la cuál marca el final de todo el documento HTML.
\end{enumerate}

\end{document}