\documentclass[12pt, letterpaper]{article}
\usepackage[utf8]{inputenc}
\usepackage{listings}

\title{¿Qué es HTML? y ¿Por qué HTML5?}
\author{Kevin Esguerra Cardona}
\date{Junio 9 del 2022}

\begin{document}
    
\maketitle

\section*{HTML}

\textbf{HTML} o \textbf{HyperTex Markup Language} es un lenguaje de marcado de hiper texto... bueno, eso es lo que dice la wikipedia. 
Yo por ahí no paso, HTML es un el lenguaje que usamos para construir la estructura d eun sitio web. Básicamente, si lo comparamos con
el cuerpo humano, HTML hace el papel de los huesos. En la actualidad vemos HTML cómo el archivo en dónde construimos los objetos de 
nuestro sitio web. Los cuales serán modificados luego por lenguajes cómo CSS3 y JavaScript. 

Cómo intuirá aprender HTML es de suma importancia cuando uno busca ser un desarrollador web. Pues una correcta comprensión de este, aunque
no aseguré un futuro exitoso, si le brindará la facilidad de construir sitio web básicos. Los cuáles, cómo ya dije, pueden ser modificados
desde otros lenguajes. 

¡Ojo! \textbf{¡HTML NO ES UN LENGUAJE DE PROGRAMACIÓN!} HTML es un lenguaje de etiquetas. Lo cual nos dice un poco sobre su naturaleza y su
curva de aprendizaje. La cuál es bastante plana. HTML funciona por medio de etiquetas o elementos. Y estos elementos, a su vez, tienen 
atributos. Piensa en un elemento cómo una sección o parte del esqueleto humano. Lo digo así, porque cómo verás más adelante existen elementos
que anidan otros elementos, los llamados elementos de diseño; y existen también elementos que no encierran a otros elementos. 

Un atributo, es una característica dada al elemento. Existen atributos obligatorios y no obligatorios, esto depende de cada elemento y su 
naturaleza. Por ejemplo, el primer elemento de diseño de todo documento HTML es justamente el elemento html. Este elemento dicta al naveador
dónde comienza y dónde termina un documento HTML. Cómo imaginará este elemento puede tener atributos, son unos cuantos, pero ahora solo vamos
conocer uno de ellos. El atributo lang. Este atributo lo que hace es definir el idioma que vamos a utilizar dentro de nuestros textos, dentro
de nuestra página web. Un ejemplo de su implementación sería: 

\begin{lstlisting}[language=HTML]
    <html lang="fr">

    </html>
\end{lstlisting}

En este ejemplo, establecemos el idioma principal de nuestro documento HTML en Francés. 

\section*{HTML5}

Supongo que se habrá dado cuenta del número que casi siempre acompaña al nombre del lenguaje HTML5. Pues, este no es simple decoración y tampoco
busca decir HTML's en plural. La realidad es que este número es un idicativo de la versión del lenguaje. Lo que quiere decir que han existido otras
4 versiones de HTML que fueron utilizadas en producción. Lanzadas al público. Pero, cómo todo, las cosas cambian. La tecnología avanza y estas
versiones se fueron quedando obsoletas. Hoy en día HTML5 es un lenguaje de marcado de hipertexto que soporta gran variedad de funcionalidades. 
Esto podemos verlo reflejado en la cantidad inmensa de etiquetas que hacen de este un lenguaje para siempre estar aprendiendo algo nuevo. 

Pero, no hay de que asustarse. HTML5 es, de hecho, bastante amigable con el programador. Tanto, que se suele usar cómo un medio de aproximación 
al momento de aprender a programar páginas web. 

\end{document}