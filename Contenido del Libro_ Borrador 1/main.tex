\documentclass[12pt, letterpaper]{article}
\usepackage[utf8]{inputenc}

\title{Contenido del Libro: Borrador 1}
\author{Kevin  Esguerra Cardona}
\date{Mayo 30 del 2022}

\begin{document}

\maketitle

\begin{enumerate}
    \item Prefacio/Introducción.
    \item Parte 1
    \begin{itemize}
        \item La historia del internet.
        \item La evolución del desarrollo web.
        \item Previas y programas necesarios (instalación). \newline*Solo se escribirá  una vez este el resto del curso completo
    \end{itemize}
    \item Parte 2
    \begin{itemize}
        \item Fundamentos de HTML5.
        \begin{enumerate}
            \item ¿Qué es HTML? y ¿Por qué HTML5?
            \item Partes de un documento HTML: <html>, <head> y <body>.
            \item Algunos componentes HTML:
            \begin{itemize}
                \item Elementos.
                \item Atributos.
                \item Encabezados.
                \item Párrafos.
                \item Citaciones.
                \item Comentarios.
                \item Enlaces.
                \item Imágenes.
                \item Favicon.
                \item Tablas.
                \item Tablas.
                \item Listas.
                \item Elementos de tipo Block e Inline.
                \item Clases.
                \item Atributo Id.
                \item Iframes.
                \item Rutas de archivos.
                \item <head></head>
                \item Elementos de diseño.
                \item Elementos para código de computadora.
                \item Elementos semánticos.
                \item Guía de estilo HTML.
                \item Conjunto de caracteres.
                \item Codificación URL en HTML.
                \item Formularios HTML.
                \item HTML Media.
                \item Recomendaciones en HTML. \newline*A lo mejor pongo esta parte dentro de HTML Media
            \end{itemize}
        \end{enumerate}
        \item Fundamentos de CSS3.
        \begin{itemize}
            \item ¿Qué es CSS? y ¿Por qué CSS3?
            \item Sintaxis.
            \item Unidades. *CSS Units
            \item Selectores.
            \item Métodos de uso. *Externo, interno e inline
            \item Comentarios.
            \item Colores.
            \item Fondos.
            \item Bordes.
            \item Margenes.
            \item Relleno. *Padding
            \item Altura y ancho.
            \item Modelo de caja.
            \item Esquema CSS. *CSS Outline
            \item Textos.
            \item Fuentes.
            \item Iconos.
            \item Enlaces.
            \item Listas.
            \item Tablas.
            \item Display.
            \item Max-width.
            \item Posicionamiento.
            \item z-index.
            \item Desbordamiento.
            \item Flotaciones.
            \item Inline-block.
            \item Alineaciones.
            \item Combinadores.
            \item Pseudo-clases.
            \item Pseudo-elementos.
            \item Opacidad.
            \item Barra de navegación. *¿?
            \item Objetos desplegables.
            \item Selectores de atributos. *Debe ir dentro de selectores
            \item Formularios.
            \item Contadores.
            \item Diseño web.
            \item Especificidad.
            \item Regla !important.
            \item Funciones matemáticas.
            \item Esquinas redondeadas.
            \item Imágenes con borde.
            \item Gradientes.
            \item Sombras.
            \item Efectos de texto.
            \item Transformaciones 2d y 3d.
            \item Transiciones.
            \item Animaciones.
            \item Tooltips.
            \item Estilos para imágenes.
            \item Reflección de imágenes.
            \item CSS Masking.
            \item Botones.
            \item Paginación.
            \item Interfaz de usuario.
            \item Variables.
            \item Media Queries.
            \item Flexbox.
            \item Grid.
            \item RWD (Responsive Web Design)
        \end{itemize}
        \item Algoritmos y estructuras de datos con JavaScript
        \item Proyecto práctico 1.
    \end{itemize}
    \item Parte 3.
    \begin{itemize}
        \item Fundamentos de la terminal de Windows (CMD).
        \item Git y Github: ¿Solo un servicio de almacenamiento?
        \item Proyecto práctico 2. *Podría incluir PHP
    \end{itemize}
    \item Parte 4
    \begin{itemize}
        \item Fundamentos de bases de datos.
        \item Python: Una breve introducción a su sintaxis y su importancia en el desarrollo web actual.
    \end{itemize}
    \item Parte 5
    \begin{itemize}
        \item Django: Introducción, historia y su utilidad actual.
        \item Proyecto práctico 3. 
    \end{itemize}
    \item Cierre del curso.
\end{enumerate}

\end{document}
