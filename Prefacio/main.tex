\documentclass{article}
\usepackage[utf8]{inputenc}

\title{Prefacio: borrador 1}
\author{Kevin  Esguerra Cardona}
\date{Junio 1 del 2022}

\begin{document}

\maketitle

\textbf{Prefacio}

La idea detrás de este libro surge debido al aumento de la automatización en al sociedad. En la actualidad miles de empleos desaparecen 
a causa de los múltiples avances en distintas ramas del desarrollo tecnológico. Un ejemplo de ello es la automatización industrial la cuál 
lentamente ha estado haciendo desaparecer la necesidad de trabajadores calificados en las fábricas. Esto se debe a los bajos costos de
mantenimiento en contraste con los altos costos que supone un conjunto de empleados preparados para una labor específica. 

Por consiguiente, se ha visto un aumento en la necesidad de personal capacitado en el desarrollo de sistemas de información. Una de las
ramas del desarrollo de software con más auge en la actualidad es el desarrollo web, ya que se prevé un cambio de tecnologías antiguas por
interfaces modernas basadas en la web. 

Atendiendo a esta problemática, se me ocurrió brindarle la posibilidad a todo aquel que desee aprender un poco sobre una rama del desarrollo
de software tan cotizada cómo lo es el desarrollo web. Este libro está pensado no solo para aquellas personas que estén cursando o hayan 
cursado estudios superiores, sino que esta orientado a todos aquellos ciudadanos que por una u otra razón deseen aprender algo que les
permitirá mejorar su calidad de vida. 

Por otro lado, hay que dejar claro de que este libro no busca ser una biblia, la idea no es detenerse al final del último 
capítulo. Es deber del aprendiz el seguir instruyéndose y mejorando cada día con el fin de mejorar su habilidades.

Por lo mencionado, cabe aclarar que en este libro pasaré de utilizar un vocabulario técnico que pueda confundir a los no iniciados, esto siempre que me sea posible.

*Este prefacio es solo un borrador y está puesto a posibles modificaciones a lo largo del proceso de creación del libro/curso.

\end{document}
